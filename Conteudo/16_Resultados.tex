\chapter{Resultados}

Só existe uma maneira verdadeiramente honesta de avaliar um trabalho científico: submetê-lo a revisão de seus pares. Por isso existe uma banca de mestrado e doutorado. Por isso cada vez mais é importante publicar seus resultados.

Na forma atual que alunos, professores e programas de pós-graduação são avaliados é impossível imaginar uma tese onde não houve uma publicação. A política correta de um orientador deve ser não permitir a defesa de uma tese que não tenha nenhum artigo publicado. Para isso dou dois motivos: se nenhum trabalho foi apresentado para a publicação, então o aluno não demonstrou interesse, se nenhum trabalho foi aceito, a tese não demonstrou capacidade.

Publicar é responsabilidade do aluno. Cabe ao orientador auxiliá-lo nessa tarefa. Claro que, dependendo da capacidade do orientador na área, ele pode ser a força motriz do artigo. Porém, é importante que o aluno tenha a experiência de conduzir a parte principal do trabalho de publicação.

Uma tese de doutorado tem uma obrigação ainda maior: publicar artigos em revista.

Publique sempre. Antes de acabar a tese, depois de acabar a tese. A única maneira de seu trabalho ficar conhecido e você ser reconhecido é por meio de publicações. Aceite ligeiros atrasos em sua tese (que não interfiram com seu prazo) se for para publicar. Publicar dará pontos em concursos públicos para professor e tornará você conhecido na comunidade.

Ao publicar, não esqueça que os autores são, pelo menos, você e seu orientador. Geralmente o aluno vem em primeiro lugar, mas algumas vezes, principalmente quando a ideia principal vem do orientador, o nome dele vem em primeiro. Publicar sem o nome do orientador é um dos maiores pecados que um aluno pode fazer contra a relação aluno/orientador na área da Computação.

A questão da publicação está se tornando cada vez mais séria no Brasil. Tanto a CAPES quanto as universidades estão avaliando seus pesquisadores principalmente em função da quantidade e qualidade das publicações.
