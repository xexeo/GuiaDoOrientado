	\chapter{Trabalhando Comigo}
    
Para trabalhar comigo as seguintes regras são obrigatórias. Aceitar ser meu orientado implica em aceitar essas regras.

\section{Artigo}

\begin{itemize}
    \item Todo aluno de mestrado deve publicar pelo menos um artigo em co-autoria comigo, possivelmente em conjunto com outros alunos.

    \item    Todo aluno de doutorado deve publicar pelo menos um artigo por ano em co-autoria comigo, possivelmente em conjunto com outros alunos.

    \item    Todo aluno de doutorado deve publicar um artigo em revista indexada sobre o seu tema de tese de doutorado em co-autoria comigo (regra da Coppe).

\end{itemize}

\section{Compartilhamento}
\begin{outline}
\1	Todo material da tese deve ser compartilhado comigo, no mínimo para questão de backup dos dados. Você pode compartilhar via \textbf{Google Drive}, \textbf{One Drive} ou \textbf{GitHub}.
\1	Todo o seu código deve estar atualizado em um projeto privado \textbf{GitHub}, compartilhado comigo. O \textbf{GitHub} fornece projetos privados gratuitos para alunos. Eu pago e posso abrir o projeto para você.
\1	Todo o seu texto \textbf{Word} deve estar compartilhado comigo em uma das seguintes ferramentas: \textbf{Google Docs}, \textbf{OneDrive}. Um diretório deve ser compartilhado comigo. Existe uma forma razoavelmente simples de manter os documentos \textbf{Word} com controle de versão no \textbf{GitHub}.
\1	Textos em \LaTeX\  devem usar o \textbf{Overleaf} ou \textbf{GitHub} e serem compartilhados comigo.
\1	Todos os seus dados devem estar em um desses ambientes de compartilhamento: \textbf{GitHub}, \textbf{Google Docs}, \textbf{OneDrive}.
\2	Dados muito grandes devem ser combinados a parte
\2	Melhor ainda se estiverem em todos! Por exemplo, trabalhem no \textbf{Google Docs}, mantenham versões no GitHub e backups de curto no Dropbox.
\1	O status da tese pode ser mantido no \textbf{Trello} ou no próprio \textbf{GitHub}.
\2 Eu disponibilizo um estilo que permite manter o status das tarefas em um documento \LaTeX.
\1	Pesquisas bibliográficas devem ser documentadas no \textbf{Parsif.al} ou em algum documento compartilhado comigo.
\end{outline}

Ou seja, o trabalho todo deve ser compartilhado comigo, por dois motivos:
\begin{itemize}
    \item Os direitos patrimoniais de todo trabalho pertence à UFRJ, não a nós. 
    \item Segurança dos dados. Vários alunos já perderam tudo que fizeram porque seu computador quebrou ou evento semelhante.
\end{itemize}

\section{Entender o que falo, fazer o que digo}

Eu espero que meus alunos entendam o que estou falando. Para isso eles têm que fazer perguntas até entender o que quero. Estou disposto a explicar muitas vezes e de jeito diferente de cada vez. Eu falo rápido, sei muita coisa, uso referências pessoais e científicas de várias épocas, posso achar que falei tudo em uma palavra, e você não entender nada. Pergunte! O máximo que pode acontecer é, em vez de uma resposta de má qualidade, eu mandar você ler um texto de boa qualidade.

Por outro lado, se eu peço fazer algo, não espero precisar explicando cada passo. Principalmente burocracia e coisas que estão em fácil acesso ou uma busca no Google de distância. Se eu digo ``Vá à secretaria'', não me pergunte ``onde é a secretaria?''. Se eu digo ``Submeta a tese no registro da Coppe'', não me pergunte como. São coisas simples que, na verdade, você deveria saber, e estão disponíveis em vários sites e documentos. Além disso, elas mudam com o tempo, e com frequência, logo a informação que eu tenho pode não ser as corretas. Eu espero que você leia e compreenda o texto ``Mensagem a Garcia'', Anexo \ref{chap:garcia}, e preste bastante atenção à sua mensagem. Como orientador, eu quero também tornar você uma pessoa capaz de resolver problemas de forma independente.

Especificamente, eu espero que você, ao longo do tempo, saiba sobre como funciona a vida do aluno e a organização. Principalmente quanto às informações que são passadas explicitamente a você em palestras e documentos na recepção do aluno.

Além disso, eu espero que você me responda rapidamente a perguntas como: qual é o órgão financiador da sua bolsa, qual seu prazo, qual seu DRE, que cadeiras fez, qual sua nota na cadeira, etc. Se você não é capaz de responder essas perguntas básicas, eu realmente me decepciono.


