\chapter{Este Texto}

Este texto vem direto do campo de batalha para você.
Não é um texto sobre metodologia científica. Não vou ficar ensinando normas ou listando as regras de acentuação em português. Não vou ensinar um método preciso. Não vou escrever aqui uma receita de bolo, mas sim dar uma fotografia geral do que é importante, e do que não é importante, para alcançar o objetivo: defender a dissertação ou a tese e ser aprovado.

Este texto é voltado para os \textbf{meus alunos}, mestrandos e doutorandos em Engenharia de Dados e Conhecimento (EDC) do Programa de Engenharia de Sistemas e Computação (PESC) da COPPE/UFRJ. Muitas das regras explicadas aqui são do PESC, da COPPE ou da UFRJ, mas existem regras semelhantes em todos os programas de pós-graduação.

Outros alunos podem viver realidades diferentes, mas certos princípios básicos sempre serão mantidos.

De agora em diante vou usar apenas o termo tese, querendo dizer tanto uma dissertação de mestrado quanto uma tese de doutorado. Alunos de projeto final ou trabalho de conclusão de curso podem também se aproveitar deste texto.

Se você não é meu aluno, espero que possa também aproveitar um pouco da minha visão. 

