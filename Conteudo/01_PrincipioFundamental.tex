\chapter{Princípio Fundamental do Orientado}
\label{chap:pfo}
\index{Princípio Fundamental do Orientado}


\gxatencao{O princípio fundamental do orientado é que ele é o único responsável pela sua tese ou dissertação.}

Na tese, você faz um trabalho basicamente auto-ditata, independente, e, mesmo que dentro de um contexto de grupo ou laboratório, basicamente isolado. Isso é quase um requisito essencial, porque deve ser um trabalho \textbf{seu}. O importante é que seja a sua contribuição individual seja esclarecida.




\section{E o Orientador, Não Está no Mesmo Barco?}

Imagine que você está em uma regata. Você é o timoneiro do barco, tomando decisões em tempo real, segurando o leme, sentindo o vento, ajustando as velas, enfrentando as ondas e fazendo escolhas estratégicas a cada momento. 
No barco, está também o seu técnico — ele não toma o leme, mas está ao seu lado, com experiência acumulada, lendo as cartas náuticas, analisando o céu e a correnteza, sugerindo mudanças de rumo, comentando os ajustes de vela, apontando riscos à frente que talvez você ainda não tenha notado.

Você tem que fazer o barco andar. É você quem está sob o crivo do cronômetro, das regras e da linha de chegada. 
Mas o técnico está comprometido com seu sucesso, torcendo e intervindo quando pode, sempre que você permite ou pede ajuda. A comunicação é constante, mas a responsabilidade última é sua. Se ganhar, a vitória é sua. Se perder, também. Mas o trajeto foi feito em conjunto.

Essa é a relação ideal entre orientador e orientando: parceria, apoio e experiência compartilhada — sem que o orientador substitua o protagonismo e o esforço de quem está na direção da própria pesquisa.

Muitos alunos, porém, escolhem deixar o técnico em terra, ou na cabine do barco. Aí ele não pode falar, orientar, ajudar. Quando o aluno resolver chamar ou passar um rádio para o técnico, já pode estar em águas difíceis ou em má posição na regata. 

\section{O Que Faz o Orientador}

O orientador já fez a sua tese, já passou em um concurso para professor e está aí para ajudar o orientado, mas a responsabilidade final com esforço, qualidade e prazos é do candidato ao título.

Durante o desenvolvimento da dissertação ou tese, o orientador é um guia que, dependendo do perfil, das disponibilidades ou do relacionamento que cria com o orientado, pode interferir mais ou menos no trabalho deste último, fornecer mais ou menos recursos, dar a ideia fundamental ou não, porém nunca será o responsável por realizá-lo.

Devemos lembrar da piada do porco e da galinha. 
\begin{quote}
    Uma galinha e um porco resolvem abrir um restaurante juntos.
A galinha sugere: ``Podemos chamá-lo de ovos com bacon!''
O porco responde: ``Melhor não. Nesse restaurante, você está envolvida, mas eu estou comprometido.''
\end{quote}
O orientado, definitivamente, é o porco (bacon).


\section{Informações Essenciais para o Aluno}

Para seguir este princípio, o aluno, e candidato ao título, deve ter a consciência de todas as suas obrigações e direitos; para isso, deve, logo ao entrar no curso, encontrar e ler:

\begin{outline}
\1	O regulamento do seu curso.
\1	As decisões tomadas após o regulamento e que são válidas para o seu curso
\1	Todos os seus prazos, incluindo
\2	A duração da bolsa
\2	A data esperada pelo programa de pós-graduação e data máxima permitida para:
\3	Terminar os créditos.
\3	Defender os exames de qualificação, seminários de qualificação ou similares.
\3	Apresentar e defender a dissertação ou tese.
\1	Os contratos e documentos que assina, principalmente em relação à bolsa, e as responsabilidades que dela advêm.
\1	Conhecer as notas necessárias para poder concluir o curso
\2	A média necessária para defender tese.
\end{outline}




\CoppeWay{Regras da Coppe}{
No momento as principais informações relevantes para os alunos da Coppe podem ser encontradas no site da Coppe, na seção \textbf{Espaço do Aluno}~\furl{https://coppe.ufrj.br/espaco-do-aluno/}. Lá está o
a regulamentação original\furl{https://coppe.ufrj.br/wp-content/uploads/2024/06/Alunos_a_partir_2017.1.pdf} e ainda vários documentos e decisões adicionais.

Em especial, para defender tese na Coppe você precisa de nota maior que 2,0 (B) e ter no máximo um D.
}

