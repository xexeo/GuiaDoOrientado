\chapter{O Orientador}


A função do orientador é orientar, mostrar caminhos, estimulá-lo a pesquisa, gerar problemas que você possa resolver. Ele também deve ajudar com a burocracia e com problemas relacionados à universidade. 


Não é função do orientador resolver os problemas da sua tese. Porém, ele pode, eventualmente, dar contribuições essenciais.


Alguns orientadores vão ajudá-lo a resolver problemas pessoais, provavelmente apenas com conselhos, mas não é essa sua função. Ele deve conhecer os problemas para entender a sua produtividade, mas não é sua obrigação resolvê-los. Se o faz, faz por vontade própria e solidariedade.


A relação orientado/orientador é muito variada, porém deve ser sempre cordial. Faça todos os esforços possíveis para não iniciar uma discussão pessoal com seu orientador. 


\gxatencao{O respeito é essencial}


A primeira coisa a entender é que o orientador não é orientador por ser mais inteligente que você, mas por ter mais experiência que você em uma área específica. Muitas vezes os orientadores são mais novos que os orientados e mesmo assim alcançam bons resultados. É claro que é importante que você respeite a inteligência do seu orientador, mas esse não é o fator de diferença entre orientado e orientador.


\section{Dificuldades comuns com os orientadores}


Certamente seu orientador o tratará com respeito, porém os seguintes problemas podem aparecer:

\begin{outline}
\1	O orientador não tem tempo para você


\1	O orientador não leu o que você escreveu


\1	O orientador não entende o que você faz
\end{outline}

\subsection{O Orientador sem tempo}


É comum um orientador ter pouco tempo disponível. Ele tem que dar aulas, participar de reuniões, orientar outros alunos e realizar atividades como pesquisa e consultoria. Você deve se adaptar ao tempo disponível de seu orientador. 


Algumas desculpas são inaceitáveis pelo orientador e entre elas está que “você não pode sair de seu trabalho nessa hora”. Afinal, você está ou não fazendo uma tese? É possível, porém, que seu orientador concorde em o orientar na casa dele, após o expediente, e, em casos excepcionais, no fim de semana. Não é o meu caso e acredito que, tirando condições excepcionais, não devia ser o caso de nenhum orientador.


Caso seu orientador tenha problemas graves na agenda, tente marcar um almoço com ele. Outra opção é conseguir um coorientador, outro professor da mesma área ou um aluno de doutorado, caso você seja aluno de mestrado. Muitos orientadores gostam de trabalhar no regime de dupla orientação.


\subsection{	O orientador que não leu o que você escreveu}


Se o orientador não leu o que você escreveu é porque não teve tempo ou esqueceu. Se ao encontrar seu orientador ele não tiver lido o que você escreveu, faça um resumo mostrando o texto para ele. 


\gxatencao{Sempre leve uma cópia do trabalho que você está fazendo para as reuniões. }


Aparecer em uma reunião sem isso é demonstrar desinteresse pela reunião, falta de preparação ou, pior, que você não fez nada.


Muitos orientadores leem um texto por alto e ficam com uma ideia muito clara do que foi feito. Principalmente nos casos de revisão bibliográfica e resultados demonstrados em gráficos, o orientador pode em 10 ou 15 minutos ter uma ideia clara do seu trabalho e contribuir para o seu desenvolvimento. 


É importante entender que o esforço do orientador é muito relacionado ao esforço do aluno. Alguns alunos parecem frustrados porque o orientador, na segundo ou terceira vez que eles aparecem com o trabalho no mesmo estágio, não estão dispostos a conversar por muito tempo ou dar ideias. Essa reação não é nada surpreendente, se você não faz o seu trabalho, o orientador não tem como fazer o dele.


\gxatencao{Um processo de orientação se baseia na evolução do trabalho do orientado.}


\subsection{O orientador que não entende o que você faz}


Das três situações que citei, essa é a mais difícil de resolver.


Não é raro que um aluno desenvolva um assunto de tese que foge dos conhecimentos do orientador. Nesse caso muitos orientadores coíbem o desenvolvimento da tese, enquanto outros buscam soluções de compromisso. Outros podem entender como uma oportunidade de abordar novas áreas. Na verdade, dependendo do assunto, a disposição do orientador pode mudar.


Uma coorientação pode, novamente, ser uma boa solução. Uma revisão bibliográfica também pode dar ao orientador as ferramentas necessárias para auxiliar no seu trabalho. Em geral, um orientador fica feliz se seu aluno sabe mais que ele sobre um assunto.


\section{Mais sobre os orientadores}


Os orientadores também passam por ciclos de alta imaginação e excessiva realidade. Muitas conversas serão “viagens” e outras serão “convocações para pôr o pé no chão”. De certa forma, essa é a tarefa do orientador. Se você estiver viajando pouco, ele vai tentar desenvolver os limites da sua imaginação, se você estiver passeando no espaço sideral, ele vai tentar trazê-lo de volta para a realidade.


As conversas com o orientador devem ser plenamente documentadas, pelo aluno, mesmo que o orientador faça isso. Se os dois estiverem documentando a conversação, compare as notas no final. 


\gxatencao{O aluno deve sair de cada conversa com uma lista de itens a fazer}


Se seu orientador não criar essa lista, pergunte diretamente quais devem ser seus próximos passos ou sugira você mesmo uma lista de ações. 


Entre duas sessões de orientação, seu orientador certamente mudará de ideia. 


\gxatencao{Nunca jogue fora um trabalho anteriormente descartado}


Caso um assunto previamente descartado como “ruim” seja considerado “bom” em uma reunião posterior, verifique em suas anotações o motivo da decisão anterior e discuta-os com o orientador. Porém, tente não questionar a mudança de opinião, pois isso só vai levar a um sermão sobre a necessidade de se ter uma mente aberta e pronta para mudanças. Revise os defeitos e qualidades da opção sendo estudada e tome uma nova decisão.


Existem muitos tipos de orientadores: o viajante, o amigão, o carrasco, o executivo, o objetivo etc. Todos esses têm suas vantagens e desvantagens, cabe a você descobrir quais os defeitos do seu orientador e evitar que eles tenham má influência na sua tese. Se seu orientador tem muitas ideias, você deve ser objetivo, se seu orientador é relapso com prazos, você deve cumprir todos. Orientadores são seres humanos e têm defeitos, muitas vezes graves. 


Só você pode evitar que esses defeitos influenciem na sua tese. Quanto aos seus defeitos, fique certo de que o orientador irá apontá-los no decorrer do relacionamento, algumas vezes até de maneira um tanto rude.


\section{Confiança no orientador}


O maior desejo do orientador é que o aluno termine a tese. É quase inconcebível imaginar que um orientador não deseje que o aluno complete o mais rápido possível e da melhor forma, o seu trabalho. 


Por que falo isso com tanta certeza? Porque os professores são avaliados, parcialmente, pela capacidade de fazer seus alunos defenderem suas teses e publicarem artigos sobre elas.


Alguns alunos, porém, imaginam que o professor está contra eles. Acham que estão pedindo trabalhos impossíveis, apenas para o benefício próprio, ou pior, para atrapalhá-los.


A verdade é que cada orientador determina um nível de qualidade aceitável para o trabalho do aluno. Esse nível é compatível com as características do aluno. 
Assim, um aluno capaz de fazer ótimos programas de computador, mas péssimo em teoria, é estimulado, e cobrado, a explorar suas qualidades ao máximo e a lutar, na medida do possível e do aceitável, contra suas dificuldades.


O orientador trabalha \textbf{sempre a favor do aluno}, dentro de algumas restrições pessoais e institucionais. 
Essas restrições envolvem a área de pesquisa, a qualidade do resultado, a dedicação ao trabalho e muitos outros fatores. 
Poucas vezes um orientador reprova ou abandona um aluno.
 Ele sempre tenta ao máximo encontrar um caminho de sucesso. 
 Essa é a tarefa principal do orientador.


Uma restrição importante que todo aluno deve estar atento é que a carreira acadêmica do orientador é fortemente, se não unicamente, influenciada pela quantidade e qualidade de suas publicações. 
O orientador é um professor que tem que arcar com muitas responsabilidades: aulas, administração da faculdade, orientação e escrever artigos. 
Assim, os orientadores normalmente esperam que o orientado os auxiliem na tarefa de escrever artigos. 


Os alunos que desejam seguir carreira acadêmica devem ficar especialmente preocupados em publicar, afinal, eles também serão julgados por sua capacidade de produção de artigos.


\gxatencao{Assim, o aluno deve estabelecer uma relação de confiança e colaboração com seu orientador.}


\section{A Escolha}


A escolha do orientador é um processo bastante complicado. Alguns alunos não têm essa opção, pois são selecionados desde o início para serem orientados por um professor. 


Você pode analisar um orientador de acordo com as seguintes facetas:

\begin{enumerate}
	\item 	Compatibilidade pessoal


	\item 	Assuntos comuns


	\item 	Qualidade como orientador


	\item 	Qualidade como professor


	\item 	Qualidade como pesquisador


	\item 	Opinião pessoal

\end{enumerate}

Existem muitas maneiras de iniciar essa seleção. Normalmente você deve ter como opção os professores com qual já fez alguma cadeira ou trabalho. Seus colegas mais antigos também são capazes de dar informações sobre os professores disponíveis. Além disso, muitas vezes outros professores podem recomendar um colega, de acordo com seus objetivos como aluno.


Um conselho importante é não se assustar muito com orientadores com fama de durões. Um orientador durão pode ser um ótimo orientador e manter você nos “trilhos certos”. Cuidado, porém, com os que tem fama de mal-educados.


\section{Problemas com o orientador}


Aconteceu! Você discutiu com seu orientador de forma irreconciliável ou se acha prejudicado fortemente. O que fazer?


Primeiro, respire. É importante parar para pensar, pois existe um registro histórico que é desfavorável a você: seu orientador já orientou diversos alunos e trabalhos, você não fez nenhuma tese.


É fato que todo relacionamento de longo prazo está sujeito a turbulências. Namoros e casamentos acabam, por que uma orientação não pode acabar? O problema é que, nesse caso, normalmente apenas um lado é prejudicado: o aluno.


Para resolver o problema temos que pensar em um escalonamento de soluções. 


Por incrível que pareça, a primeira pessoa que pode ajudá-lo é o próprio orientador. Tente marcar outra reunião e de forma educada dizer que não consegue mais trabalhar nas condições atuais. Lembre-se que, sendo a parte mais frágil, acusar pouco vai servir a você. Seu orientador então pode se propor a buscar um novo orientador ou um coorientador. Essa solução é a mais fácil.


A seguir, caso isso não funcione, você deve caminhar lentamente pelas instâncias superiores da instituição. Na COPPE existe um chefe de linha, a coordenação acadêmica e o coordenador, dentro do Programa. No nível de diretoria ainda existe o Coordenador Acadêmico e o Conselho de Pós-Graduação e Pesquisa(CPGP).



