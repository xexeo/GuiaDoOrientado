\chapter{O Ótimo é Inimigo do Bom}

Se eu tivesse que resumir tudo em uma frase, escolheria 

\gxatencao{O ótimo é inimigo do bom.}


Vamos ver três objetivos durante o curso de pós-graduação:
\begin{itemize}
\item	Você tem que acabar a sua tese. 
\item	Acabar a tese é a coisa mais importante. 
\item	Nada é mais importante que acabar a tese.
\end{itemize}

Está claro? 
A tese é o mais importante!

Para que você possa acabar sua tese, ela tem que estar bem delimitada. Isso não precisa ser feito no início de tudo, mas pelo menos seis meses antes do seu prazo terminar, a tese tem que estar totalmente delimitada. 

Uma tese também não é a palavra final sobre o assunto. É muito pouco provável que você tenha um resultado em sua tese que mude a história da ciência ou o dia a dia das pessoas. 

Sua tese é uma contribuição ao conjunto de trabalhos existentes. Se você acha que sua tese mudará tudo, é bom ter uma conversa séria com seu orientador. 

Quando digo séria, digo uma conversa em que você vai extremamente preparado para provar sua tese ou a possibilidade dela e que levará argumentos e contra-argumentos que apoiem sua previsão. Para falar a verdade, é importante fazer isso qualquer que seja sua tese.

É importante cortar. Sempre queremos escrever mais do que devemos, queremos estudar um pouco mais, queremos entender um pouco melhor aquele detalhe. Porém, o que é realmente importante? Acabar a tese. 

Se a tese só crescer, ela nunca terá fim. Cortando, geramos o foco que permitirá que a tese chegue ao seu bom termo.
Você não precisa ser inteligente para acabar a tese. Precisa ser dedicado e trabalhar muito. Porém, sempre que tomar uma decisão lembre-se dessas regras simples, sugeridas por orientadores e orientados:
\begin{itemize}
    \item A tese tem que acabar. 
    \item O ótimo é inimigo do bom.
    \item Você é humano.
   \item 	Apenas com trabalho se alcançam resultados.
\item	Entre um conjunto de soluções, a mais simples deve ser a primeira a ser tentada.
\item	Quem não se comunica se trumbica
\end{itemize}
	

A experiência diz que os alunos quebram todas essas regras em vários momentos do processo de construção da tese. Eles querem fazer coisas demais, querem ser perfeitos demais, assumem compromissos demais, esperam resultados de graça e tentam soluções complicadas antes de tentar as fáceis. Não quebrando essas seis regras, você facilitará muito o seu trabalho e o do seu orientador.

Também é importante limitar o número de assuntos a serem estudados profundamente. O número ideal é um, mas dois é um número possível. Tentar tratar profundamente de três assuntos está definitivamente fora de questão. Se sua tese é multidisciplinar, escolha assuntos principais e trate os outros como ferramentas ou área de referência.

Para sua tese acabar, você ou seu orientador têm que tomar uma decisão importante: é o fim. Geralmente, em uma reunião se combinam todos os pontos a serem costurados, terminados e abandonados para a tese terminar. A partir desse ponto, o orientador espera apenas verificar o seu texto e resultados, mas não ter nenhuma novidade no processo. Alcançar esse momento é um sinal importante de que você irá defender sua tese logo.

\begin{itemize}
    \item Escreva uma sentença de até 25 palavras sobre o tema da sua tese.
    \item Não misture mais de 2 áreas de pesquisa.
    \item Encontre primeiro o problema, depois a solução.
\end{itemize}
