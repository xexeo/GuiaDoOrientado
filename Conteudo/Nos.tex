\chapter{Nós}

Nesse capítulo discuto os principais envolvidos em um projeto de tese: o orientando e o orientado.

\section{Você – O Orientado}

Você deve ser muito inteligente, já que se candidatou e foi aceito para a pós graduação. Deve ter também um bom currículo e estar acostumado a ter sucesso em sua vida acadêmica e, se não for recém formado, na profissional.
Nada disso será o fator determinante para você acabar sua tese e obter seu título.

Para defender sua tese você precisa escrever e, em alguns casos, construir um artefato, fazer um protótipo, executar experimentos computacionais ou uma pesquisa de campo.

Isso se resume a trabalhar com dedicação. A inteligência pode ajudar e até mesmo diminuir seu esforço, mas é a dedicação que fará com que você alcance seus objetivos.

Vários alunos inteligentes não conseguem obter seus títulos. Isso acontece porque eles caem em várias armadilhas, muitas causadas pela própria inteligência. Porém é raro ver um aluno dedicado que não defenda sua tese, inclusive com brilhantismo.

Esse texto pretende indicar alguns caminhos, apontar algumas armadilhas e auxiliá-lo, de várias formas, a se preparar para essa difícil tarefa.

\section{Eu - O Orientador}

Eu sou professor de graduação e pós-graduação, doutor, engenheiro e orientador em teses, dissertações e projetos finais. Para chegar a essa posição eu também tive que defender um projeto final e uma tese\footnote{Eu nunca fiz um mestrado!}. Vi também amigos meus passarem por essa experiência. Acompanho meus alunos e alunos de outros professores.

\section{Este Texto}

Este texto vem direto do campo de batalha para você.
Não é um texto sobre metodologia científica. Não vou ficar ensinando normas ou listando as regras de acentuação em português. Não vou ensinar um método preciso. Não vou escrever aqui uma receita de bolo, mas sim dar uma fotografia geral do que é importante, e do que não é importante, para alcançar o objetivo: defender a dissertação ou a tese e ser aprovado.

Este texto é voltado para os meus alunos, mestrandos e doutorandos em Engenharia de Dados e Conhecimento do Programa de Engenharia de Sistemas e Computação da COPPE/UFRJ. Outros alunos podem viver realidades diferentes, mas certos princípios básicos sempre serão mantidos.

De agora em diante vou usar apenas o termo tese, querendo dizer tanto uma dissertação de mestrado quanto uma tese de doutorado. Alunos de projeto final ou trabalho de conclusão de curso podem também se aproveitar deste texto.

Se você não é meu aluno espero que possa também aproveitar um pouco da minha visão.

