\chapter{Princípio Fundamental do Orientado}
\label{chap:pfo}
\index{Princípio Fundamental do Orientado}


\gxatencao{O princípio fundamental do orientado é que ele é o único responsável pela sua tese ou dissertação.}

O orientador já fez a sua tese, já passou em um concurso para professor e está aí para ajudar o orientado, mas a responsabilidade final com esforço, qualidade e prazos é do candidato ao título.

Durante o desenvolvimento da dissertação ou tese, o orientador é um guia que, dependendo do perfil, das disponibilidades, ou do relacionamento que cria com o orientado, pode interferir mais ou menos no trabalho deste último, fornecer mais ou menos recursos, porém nunca será o responsável por realizá-lo.

Para seguir este princípio, o aluno, e candidato ao título, deve ter a consciência de todas as suas obrigações e direitos, para isso deve, logo ao entrar no curso, encontrar e ler:

\begin{outline}
\1	O regulamento do seu curso.
\2	No momento o \gxdefine{regulamento da COPPE} pode ser encontrado em \url{https://coppe.ufrj.br/sites/default/files/arquivo_cpgp/Alunos_a_partir_2017.1.pdf }
\1	As decisões tomadas após o regulamento e que são válidas para o seu curso
\2	Na COPPE, aparecem em \url{https://coppe.ufrj.br/pt-br/node/3464}
\1	Todos os seus prazos, incluindo
\2	A duração da bolsa
\2	A data esperada pelo programa de pós-graduação e data máxima permitida para:
\3	Terminar os créditos.
\3	Defender os exames de qualificação, seminários de qualificação ou similares.
\3	Apresentar e defender a dissertação ou tese.
\1	Os contratos e documentos que assina, principalmente em relação a bolsa, e as responsabilidade que dela advém.
\1	Conhecer as notas necessárias para apresentar suas defesas
\2	Na COPPE, a média deve ser B (2,0) ou maior.
\end{outline}
