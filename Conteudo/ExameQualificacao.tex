\chapter{O Exame de Qualificação}

Praticamente em todos os programas de doutorado o aluno deve ser aprovado em um exame de qualificação para se tornar um candidato ao doutorado.

O exame de qualificação busca avaliar 3 quesitos:

\begin{enumerate}
\item	O aluno tem conhecimento suficiente para realizar o doutorado?
\item	O tema de tese conterá uma contribuição original?
\item	O tema de tese é factível?
\end{enumerate}

Um cuidado a ser tomado é não pensar que o exame de qualificação é como uma tese. Como documento, ele deve ser bem menor, como assunto, ele deve ser focado na proposição e na avaliação de viabilidade.

Como há um prazo máximo, os exames podem ser feitos em vários momentos da tese. Exames mais tardios exigem que o aluno já tenha trabalhado razoavelmente no tema. Há locais onde para o exame ser feito a tese já tem que estar toda estabelecida e só estejam faltando coisas como o experimento comprobatório.

Um exame feito mais cedo pode ser mais teórico, com menos contribuições já realizadas.
Para mim, um exame de qualificação deve ter pelo menos uma tentativa de solucionar o problema, ou uma investigação na dificuldade de fazê-lo, por meio de tentativas mais ou menos sofisticadas, dependendo do tempo que já foi gasto do prazo da tese.
