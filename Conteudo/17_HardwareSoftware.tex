\chapter{Hardware e Software}

Quando comecei a escrever esse guia, ainda havia alunos que chegavam na pós-graduação e não tinham um computador conectado na Internet. Nem posso imaginar que alguém esteja nessa condição hoje em dia.

Porém, seu computador está apto a atender as exigências da sua tese? 

Se necessário, compre um computador novo. Basicamente, verifique que computadores as pessoas do seu grupo utilizam. PCs são mais comuns do que Macs no Brasil e geralmente são utilizados nas áreas de engenharia.




\section{Aplicativos Básicos}

Durante a tese você precisará certamente de um editor de textos, um editor de apresentações e, na maioria esmagadora dos casos de pesquisa, uma planilha eletrônica. O Microsoft Office é uma excelente opção. Muitos alunos conseguem fazer tudo usando apenas os aplicativos da Google, mas é quase impossível formatar uma tese corretamente no Google Docs.

Não gosto do Open Office Write e outras ferramentas livres que tentam imitar o Word e Excel , oferecendo funcionalidades similares, porém em geral de pior qualidade e com sérios problemas de compatibilidade até entre si. Não recomendo o uso dessas ferramentas para o texto da tese.

\subsection{Planilha}

Muitas teses precisam de apresentar alguns resultados na forma de tabelas e gráficos. O programa de escolha genérico é, com larga margem, o Microsoft Excel, com o Google Sheet vindo em segundo lugar. 


Existem, porém, programas melhores que as planilhas para fazer gráficos a partir de números. Gnuplot é um programa livre que, com conhecimento, pode produzir gráficos de alta qualidade. Isso se pode dizer de vários problemas de manipulação matemática (como o Matlab) ou estatística (como o SPSS).
Existem outros software, como o Tableau, que também podem ajudar.


\subsection{\LaTeX}

Atualmente, e isso já variou com o tempo, meu software de escolha para escrever teses e artigos é o \LaTeX, por meio do ambiente colaborativo Overleaf. 

O \LaTeX pode também ser usado em sua máquina. Existem versões open-source e gratuitas do TeX e LaTeX, como o MikTeX, para Windows. 

O importante é que, com qualquer das duas soluçoes, mantenha o versionamento, e consequentemente um backup, no GitHub (ou outra ferramente semelhante).

Recomendo também não manter sua tese em um único arquivo. Arquivos longos tendem a criar problemas de edição. No LaTeX é possível quebrar um arquivo e usar comandos de inclusão em um arquivo principal, o que facilita o trabalho. 

\section{BACKUP!}

Backup deve ser seu deus! 
Vários, todos os dias em vários formatos. Guarde seu trabalho com amigos e com seu orientador. Faça backup dos backups. Em qualquer acidente, o backup o salvará. 

Gaste dinheiro com backup. Não reutilize discos de backup. Se possível, tenha um método para guardar grandes quantidades de informação. Tenha um disco rígido externo. 
Existem vários softwares e serviços de backup disponíveis. 

Outra forma é enviar arquivos para backup em sua conta Google ou outra conta criada especialmente para isso.

Mantenha um backup atualizado, de preferência de forma automática.


\section{Referências}

Use um software de controle bibliográfico. Existem vários no mercado, alguns gratuitos. Eu aconselho o Zotero, que é gratuito, funciona em rede, e é adotado pela linha de Engenharia de Dados e Conhecimento. Outro bom software gratuito é o Mendeley. 
Os usuários de LaTeX podem usar o JabRef.

Anote tudo que ler. Faça fichamento ou coloque no software de controle bibliográfico. 
O ideal é que você não tenha que ler nada duas vezes (a não ser na primeira vez, que pode na verdade exigir várias leituras). Mantenha o resumo de tudo. 

\section{Desenhando}

Sua tese deve usar desenhos, os softwares livres são muito bons. O PowerPoint também pode ser usado para fazer desenhos bem decentes.

\section{Programas matemáticos}

Caso vá fazer algum trabalho com matemática, mesmo que vá desenvolver na tese programas próprios, é importante ter um software de referência na área, como MatLab, WolfranResearch Mathematica e MathCad, dependendo das ferramentas que precise.
Programas livres: Scilab (substitui o Matlab). 

Outra opção interessante é usar a linguagem Python e os ambientes matemáticos como Anaconda.

\section{Alternativas}

Muitas vezes é desejável usar um software específico, porém ele é muito caro. Os alunos devem investigar:

\begin{itemize}
    \item Versões de aluguel mensal
    \item Versões acadêmicas
    \item Versões específicas para alunos
    \item Opções mais baratas
    \item Opçoes de software livre
\end{itemize}

Curiosamente, alguns softwares já tiveram uma versão gratuita, ou de comunidade, que ainda pode ser encontrada na rede. Elas não têm mais manutenção. Um caso desse tipo é o Astah. 


{\small
\begin{longtable}{
  >{\raggedright\arraybackslash}p{0.28\linewidth}
  >{\raggedright\arraybackslash}p{0.44\linewidth}
  >{\raggedright\arraybackslash}p{0.24\linewidth}
}
\toprule
\textbf{Software Pago} & \textbf{Alternativas Gratuitas / Livres} & \textbf{Categoria de Uso} \\
\midrule
\endfirsthead

\toprule
\textbf{Software Pago} & \textbf{Alternativas Gratuitas / Livres} & \textbf{Categoria de Uso} \\
\midrule
\endhead

MAXQDA & Taguette, RQDA, QualCoder & Análise Qualitativa de Dados \\
NVivo & CATMA, TAMS Analyzer, Taguette & Análise Qualitativa de Dados \\
ATLAS.ti & Cassandre, QDA Miner Lite & Análise Qualitativa de Dados \\

MATLAB & GNU Octave, Scilab, SageMath & Computação Científica \\
Mathematica & SageMath, Maxima, SymPy & Computação Simbólica \\
Maple & SymPy, SageMath, wxMaxima & Computação Simbólica \\

Tableau Desktop & Google Data Studio, Metabase, Grafana & Visualização e BI \\
Power BI Pro & Apache Superset, Metabase, Redash & Visualização e BI \\
Qlik Sense & KNIME, Orange & Visualização e BI \\

Microsoft Office & LibreOffice, OnlyOffice, Google Docs & Escritório / Produtividade \\
Microsoft Word & LibreOffice Writer, AbiWord, Google Docs & Processamento de Texto \\
Microsoft Excel & LibreOffice Calc, Gnumeric, Google Sheets & Planilhas \\
Microsoft PowerPoint & LibreOffice Impress, Google Slides & Apresentações \\

EndNote & Zotero, JabRef, Mendeley & Gerência Bibliográfica \\
RefWorks & Zotero, BibDesk, JabRef & Gerência Bibliográfica \\

Intel Fortran Compiler & GFortran & Programação Científica \\
Matlab IDE & GNU Octave, VS Code + Extensão & Programação Científica \\
Visual Studio Enterprise & Visual Studio Community, VS Code & IDE / Desenvolvimento \\

SPSS & PSPP, Jamovi, JASP, R & Estatística / Ciências Sociais \\
SAS & R, Python (pandas, statsmodels) & Estatística / Dados \\
GraphPad Prism & SciDAVis, Veusz, R + ggplot2 & Estatística / Visualização \\

\bottomrule
\end{longtable}
}

