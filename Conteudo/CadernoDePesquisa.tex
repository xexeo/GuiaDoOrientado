\chapter{O Caderno de Pesquisa}

Uma prática que dá muito certo é os alunos manterem um caderno normal, de papel mesmo, como registro de trabalho e reuniões.

Tenha sempre o caderno ao conversar com o orientador.

Anote no caderno tudo que você fez, tudo que vai fazer.

Sei que muitos pensam em fazer isso digitalmente. Não é prático, pois não pode desenhar ou rabiscar no celular com a facilidade que faz no caderno.

Cadernos de pesquisa (ou de laboratório) são uma prática antiga que funciona. Recomendo fortemente.

Também são ótimos para criatividade e para guardar ideias que saem do nada.

Eu gosto de usar cadernos de folha branca ou quadriculada.
