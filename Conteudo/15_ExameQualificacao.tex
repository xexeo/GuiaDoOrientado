\chapter{O Exame de Qualificação}

Praticamente em todos os programas de doutorado o aluno deve ser aprovado em um exame de qualificação para se tornar um candidato ao doutorado. Já, no mestrado, essa obrigação é menos presente, porém existe, de forma simplificada, na Coppe.

\section{O Exame de Qualificação de Doutorado}

O exame de qualificação busca avaliar 3 quesitos:

\begin{enumerate}
\item	O aluno tem conhecimento suficiente para realizar o doutorado?
\item	O tema de tese conterá uma contribuição original?
\item	O tema de tese é factível?
\end{enumerate}

Um cuidado a ser tomado é não pensar que o exame de qualificação é como uma tese. Como documento, ele deve ser bem menor; como assunto, ele deve ser focado na proposição e na avaliação de viabilidade.

Como há um prazo máximo, os exames podem ser feitos em vários momentos da tese. Exames mais tardios exigem que o aluno já tenha trabalhado razoavelmente no tema. Há locais onde, para o exame ser feito, a tese já tem que estar toda estabelecida e só estejam faltando coisas como o experimento final comprobatório ou mesmo uma redação final, funcionando praticamente como um sinal verde para a defesa, ou um sinal de que falta algo importante. 

Um exame feito mais cedo pode ser mais teórico, com menos contribuições já realizadas.
Para mim, um exame de qualificação de doutorado deve ter pelo menos uma tentativa de solucionar o problema, ou uma investigação na dificuldade de fazê-lo, por meio de tentativas mais ou menos sofisticadas, dependendo do tempo que já foi gasto do prazo da tese.

No PESC é comum que o exame seja feito razoavelmente cedo, quando a proposta é caracterizada, em torno de dois anos do início da tese.

\section{O Seminário de Qualificação de Mestrado}

Esse evento, no PESC, é mais simples que o Exame de Qualificação de Doutorado. Por exemplo, não é exigida uma banca, apenas a aprovação do orientador. É comum, porém, que exista uma banca que apoia o orientador.

A minha prática é exigir um documento na forma de artigo que faça uma revisão do tema da dissertação e proponha e justifique um projeto de dissertação, com objetivos e cronograma.

\section{Um Exame para Favorecer o Candidato}

Uma característica da evento de qualificação, seja ele o seminário ou exame, é que a banca é vista como um órgão consultor, ou seja, se espera que os membros da banca dêem sugestões para que o produto final seja melhor. 

Além disso, a banca pode indicar limites, tanto mínimos quanto máximos do que espera da tese ou dissertação. É mais comum que a proposta seja muito ampla, e é costume da banca avisar ao candidato para estreitar seu foco. Raras vezes a banca espera que seja feito mais do que é proposto, mas também é comum que a banca indique caminhos alternativos, ou caminhos adicionais que o candidato pode ou deve seguir, por exemplo chamando a atenção para um artigo, método ou teoria que o candidato não demonstrou ter conhecimento.
