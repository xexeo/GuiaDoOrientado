\chapter{Publicando}

\begin{center}
\includegraphics[width=0.5\linewidth]{Images/publicando.png}    
\end{center}
\vspace{0.5cm}


Só existe uma maneira verdadeiramente honesta de avaliar um trabalho científico: submetê-lo a revisão de seus pares. 
Por isso existe uma banca de mestrado e doutorado. Por isso cada vez mais é importante publicar seus resultados.

Na forma atual que alunos, professores e programas de pós-graduação são avaliados é impossível imaginar uma tese onde não houve uma publicação.
A política correta de um orientador deve ser não permitir a defesa de uma tese que não tenha nenhum artigo publicado. Para isso dou dois motivos: se nenhum trabalho foi apresentado para a publicação, então o aluno não demonstrou interesse, se nenhum trabalho foi aceito, a tese não demonstrou capacidade.

Publicar é responsabilidade do aluno. Cabe ao orientador auxiliá-lo nessa tarefa. Claro que, dependendo da capacidade do orientador na área, ele pode ser a força motriz do artigo. Porém, é importante que o aluno tenha a experiência de conduzir a parte principal do trabalho de publicação.

Uma tese de doutorado tem uma obrigação ainda maior: publicar artigos em revista.

Publique sempre. Antes de acabar a tese, depois de acabar a tese. A única maneira de seu trabalho ficar conhecido e você ser reconhecido é por meio de publicações. Aceite ligeiros atrasos em sua tese (que não interfiram com seu prazo) se for para publicar. Publicar dará pontos em concursos públicos para professor e tornará você conhecido na comunidade.

Ao publicar, não esqueça que os autores são, pelo menos, você e seu orientador. Geralmente o aluno vem em primeiro lugar, mas algumas vezes, principalmente quando a ideia principal vem do orientador, o nome dele vem em primeiro. Publicar sem o nome do orientador é um dos maiores pecados que um aluno pode fazer contra a relação aluno/orientador na área da Computação.

A questão da publicação está se tornando cada vez mais séria no Brasil. Tanto a CAPES quanto as universidades estão avaliando seus pesquisadores principalmente em função da quantidade e qualidade das publicações.

 \needspace{5\baselineskip}\section{A Pressão para Publicar Frequentemente}

Devido às metodologias de avaliação a que estão submetidos os diversos programas de pós-graduação, publicar se tornou uma atividade imperativa ao longo de dissertações de mestrado e teses de doutorado.

Normalmente são feitas as seguintes avaliações que consideram as publicações:

\begin{itemize}
  \item Os professores são avaliados dentro de seus programas de pós-graduação, para poder orientar
  \item Os professores são avaliados para promoções
  \item Os professores são avaliados quando fazem pedidos de bolsa ou projetos
  \item Os programas são avaliados pela Capes e pelo CNPq
  \item As universidades são avaliadas pelo MEC, por organismos nacionais e internacionais 
\end{itemize}

 \needspace{5\baselineskip}\section{As Regras da Publicação}

As regras da publicação são:
\begin{enumerate}
    \item Escrever sempre;
    \item Relacione como autores todos os envolvidos;
    \item Sempre envolva seu orientador, e
    \item Escute os revisores.
\end{enumerate}

Tudo que você fizer no mestrado deve levar em conta a possibilidade de uma publicação. Se não permite uma publicação é porque provavelmente não há contribuição.

 \needspace{5\baselineskip}\section{Autoria}

Na Computação é praxe que os artigos sejam feitos com a colaboração direta dos orientadores e alunos.
Em outras áreas isso pode não ser verdade, e os textos podem ter um autor só. Na Medicina, por outro lado, é comum artigos com muitos autores, e existem até regras específicas para colocar uma pessoa como autor ou não\feurl{https://www.icmje.org/recommendations/browse/roles-and-responsibilities/defining-the-role-of-authors-and-contributors.html}{Papéis de autores e contribuidores de um artigo}.


O comportamento esperado é avisar o orientador do artigo desde o início e manter ele dentro do \textit{loop}. Além disso, nunca deixe de apresentar o artigo ao orientador antes de submetê-lo a um congresso ou revista. Só retirar o nome se o orientador pedir.  
Um orientador espera que você trabalhe com ele, e não que saia fazendo trabalhos isolados com outras pessoas. Ele também evitará que você cometa erros grosseiros por inexperiência. 

É melhor errar por incluir algum autor que não merecia estar citado do que excluir um que merecia. 
Não se esqueça de enviar o artigo a todos os autores, pedir sua colaboração e aceitação. 
Não coloque pessoas como autoras sem avisá-las.
Cabe a quem não se interessar solicitar a retirada do seu nome do artigo, e a quem se interessar, participar ativamente.

Sempre se pergunte: esse texto seria produzido, da forma como está, caso a pessoa específica não tivesse dado sua colaboração? 

\subsection{Ordem de autores}

A questão da ordem de autores é normalmente resolvida colocando em ordem decrescente de manipulação do texto. Normalmente, o autor principal é aquele que contribuiu mais para as partes mais importantes do texto. Os orientadores costumam ficar no final.

 \needspace{5\baselineskip}\section{Escolhendo onde publicar}

Para publicar são usados os seguintes critérios na escolha da revista ou congresso:
\begin{itemize}
    \item A adequação do tema ao veículo;
    \item A comunidade a que pertence o pesquisador, que pode se interessar no artigo e como o veículo a atende;
    \item O impacto do veículo na área, seja ela específica ou geral, já que algumas revistas de baixo impacto em geral tem alto impacto em uma área específica, por exemplo, porque essa área tem menos pesquisadores;
    \item A taxa média de aceitação;
    \item A quantidade de artigos publicados e a frequência de publicação;
    \item O estilo de artigos da revista, já que alguns veículos usam linguagens mais formais, ou mais matemáticas, e outros são mais informais, mesmo que ainda acadêmicos;
    \item A linguagem do veículo;
    \item A velocidade do processo de revisão e publicação, e
    \item O custo de publicação ou apresentação em congresso.
\end{itemize}


Muitas universidades têm acesso a sistemas que permitem verificar o impacto das publicações, como o \textit{Web of Science}. Além disso, quase todas as áreas já conseguiram organizar seu ``Qualis'', uma lista mantida pela CAPES classificada das publicações que tem como objetivo indicar sua qualidade. Essa lista, porém, tem falhas e foi anunciado em 2025 que não será mais mantida.

\gxatencao{O primeiro alvo de suas publicações deve ser os mesmos congressos e revistas que você usa na sua bibliografia.}

\subsubsection{Revistas e congressos predatórios}

Ao buscar onde publicar os resultados de sua pesquisa, é imprescindível estar atento às chamadas \textbf{publicações e eventos predatórios}.
Esse termo refere-se a revistas científicas e congressos que se aproveitam da pressão por publicações acadêmicas para explorar autores, muitas vezes oferecendo visibilidade em troca de pagamento, sem garantir os padrões mínimos de revisão por pares ou curadoria científica.

 Essas revistas e eventos costumam aceitar artigos com critérios mínimos (ou inexistentes) de avaliação de qualidade, desde que o autor pague uma taxa de publicação. 
 É comum que enviem convites genéricos por e-mail, muitas vezes com elogios ao seu ``excelente trabalho'', mesmo sem citar um título específico. 
 Algumas chegam a informar que seu artigo já foi ``pré-selecionado'' e que a publicação ocorrerá rapidamente, mediante o pagamento de uma taxa, geralmente com valores entre 50 e 500 dólares.

\textbf{Sinais de alerta incluem:}
\begin{itemize}
  \item Ausência de revisão por pares ou um processo extremamente rápido de aceitação;
  \item Convites genéricos para submissão enviados por e-mail em massa;
  \item Convites repetidos para diferentes trabalhos em Congresso feitos pela mesma editora;
  \item Convites para revistas com nomes muito genéricos ou para revistas com nomes que não têm relação com seu artigo;
  \item Promessa de publicação garantida ou prazos de resposta muito curtos;
  \item Taxas elevadas sem justificativa, cobradas logo na submissão;
  \item Títulos enganosos que imitam revistas renomadas;
  \item Sites com informações vagas sobre o corpo editorial, indexadores ou política de ética.
\end{itemize}

Eventos predatórios também merecem atenção. Alguns congressos internacionais, muitas vezes organizados por empresas comerciais, agregam dezenas ou centenas de trilhas temáticas desconexas, com nomes genéricos como \textit{``World Congress on Advances in Science, Engineering and Technology''}. 
Esses eventos geralmente aceitam qualquer resumo e cobram taxas de inscrição elevadas, oferecendo pouco retorno acadêmico. Frequentemente, os anais nem sequer são indexados em bases reconhecidas.

\textbf{Como se proteger:}
\begin{itemize}
  \item Consulte seu orientador ou colegas mais experientes.
  \item Verifique se a revista está indexada em bases respeitadas como \textit{Scopus}, \textit{Web of Science}, ou tem fator de impacto no \textit{Journal Citation Reports}.
  \item Consulte o site do \textit{Committee on Publication Ethics} (COPE)\footnote{\expurl{https://publicationethics.org}{Committee on Publication Ethigs}} e a lista de revistas confiáveis mantida por universidades.
  \item Verifique se a editora é associada a organizações legítimas como o \textit{Directory of Open Access Journals} (DOAJ).
  \item Desconfie de congressos com escopo excessivamente amplo e sem revisão técnica clara.
\end{itemize}

O conceito de publicação predatória foi inicialmente formalizado por Jeffrey Beall, bibliotecário da Universidade do Colorado, que manteve por anos uma lista pública de editoras e periódicos suspeitos. Apesar da lista original ter sido descontinuada, versões arquivadas e alternativas ainda são úteis como ponto de partida.

Lembre-se que publicar em revistas ou congressos predatórios pode comprometer sua credibilidade acadêmica. 
Uma publicação nesse tipo de veículo não apenas não é valorizada, como pode prejudicar sua avaliação em seleções de bolsas, concursos ou progressões acadêmicas.


 \needspace{5\baselineskip}\section{Plágio e Citações}

Basicamente, plagiar significa apresentar como seu trabalho que foi feito e já publicado por outro. 
No mundo acadêmico, o plágio é considerado uma desonestidade séria e é punido de várias formas, formais e informais, como a exclusão de um curso, a reprovação de um trabalho ou em uma cadeira, a demissão de professores e até mesmo, em alguns casos, sendo levado à justiça comum.

Isso significa que todo texto para o qual assumimos a autoria deve ser original, sob o risco de incorrer em plágio. 
Obviamente, não é possível fazer trabalho científico sem se utilizar de ideias e textos de outros autores como ponto de partida e apoio, logo existem regras claras de como realizar \textbf{citações e referências}, isto é, como descrever o trabalho de outro de forma que fique clara a atribuição de autoria.

No Brasil existe uma norma de citação mantida pela ABNT e muitas universidades mantêm versões próprias, inspiradas na ABNT. Nessas normas se descrevem, de forma bastante detalhada, as várias maneiras de se declarar uma citação. Nem sempre, porém, fica claro o que é uma citação.

Existem duas formas de citação: a citação direta e a citação indireta. 

Na citação direta copiamos diretamente o texto do autor e, por causa disso, devemos marcar de forma clara que estamos fazendo essa cópia. Segundo a norma ABNT isso é feito pelo uso de aspas, quando a citação tem até 3 linhas, ou usando um parágrafo com recuo de 4 cm da margem esquerda]\citep{abnt10520_2023}.

\begin{quote}
``citações indiretas (ou livres) são a reprodução de algumas idéias, sem que haja transcrição das palavras do autor consultado. Apesar de ser livre, deve ser fiel ao sentido do texto original. Não necessita de aspas.'' \citep{abnt10520_2023}
\end{quote}

Nos dois parágrafos acima fizemos uma citação indireta ao descrever a direta e uma citação direta ao descrever uma indireta. 

 \needspace{5\baselineskip}\section{Como trabalhar um artigo em grupo}

Apesar de facilitar o trabalho, escrever um artigo em grupo, principalmente se feito remotamente, cada um em um local diferente, não é uma tarefa fácil. O trabalho de um pode destruir o trabalho de outro, estilos podem ficar misturados, etc.

Algumas recomendações:

\begin{itemize}
  \item É possível fazer grande parte do trabalho textual em um editor on-line. Hoje é minha opção preferida, usando o \textit{Overleaf}, um ambiente para escrever em \LaTeX. As opções são: Google Docs e a versão nuvem do Microsoft Word. 
  \item Overleaf é uma boa opção para o mundo \LaTeX, porém para compartilhar entre muitas pessoas, deve ser pago. Eu pago, e costumo abrir projetos para meus alunos poderem fazer sua tese e artigos. Nem todas revistas aceitam artigos em PDF ou em formato \LaTeX.
  \item A versão final deve provavelmente ser feita em LaTeX ou Word, devido a capacidade de formatação. Para isso use serviços com Google Drive e OneDrive, compartilhando pastas. Algumas revistas aceitam vários formatos, outras apenas um.
  \item Use um sistema de controle de versão, como git e GitHub, ou, se não usar o controle de versão, use nomes para controlá-la. O nome do arquivo deve ser algo do tipo \texttt{Nome do artigo – parte – versão – autor que fez a versão.docx}.\\
  Exemplo: \texttt{theboss texto v10 xexeo.docx}. 
\end{itemize}

\subsection{Organização e versionamento}

Mantenha a última versão do artigo na raiz do seu diretório, junto com a template original da revista ou congresso e o call for papers.

Mantenha diretórios separados que indicam o que está nele, como em:
\begin{itemize}
  \item \textbf{Versões antigas}
  \item \textbf{Subsídios} (contendo cópias de todas as referências usadas e até mesmo algumas não usadas, como referência de escrita)
  \item \textbf{Dados} (todos os dados usados para escrever o artigo, na medida do razoável pelo consumo de espaço)
  \item \textbf{Programas} (todos os programas usados para processar os dados)
  \item \textbf{Resultados} (todos os resultados obtidos)
  \item \textbf{Software} (links ou software usado no processamento, na medida do razoável para utilização)
\end{itemize}

Além disso: 

\begin{itemize}
  \item Mantenha a revisão ligada no Word e no Overleaf.
  \item Sempre que mudanças forem feitas, faça uma nova versão do artigo, criando uma versão e passando as velhas para o diretório de versões antigas, ou fazendo commit.
  \item Possivelmente quebre o artigo em partes.
  \item Mantenha o formato da conferência o mais cedo possível, para ter ideia de tamanho.
  \item Se comunique via meios eletrônicos para registrar as comunicações. 
  \item Em certo ponto, congele o crescimento do artigo, para se preparar apenas para revisões e correções. 
  \item Ao congelar, peça para todos os autores fazerem sua última revisão. 
  \item Tente dar ao menos 24 horas para isso.
  \item Marque uma reunião final para fechar o arquivo e para que todos possam submeter conjuntamente ou concordar que um será responsável pela submissão.
\end{itemize}

Não se esqueça que existem softwares muito poderosos de controle de referência que serão essenciais à sua tese. Em especial, o uso do bib\LaTeX e bib\TeX no \LaTeX, com apoio do JabRef e o uso do controle de referências com Zotero instalado no Word são meus preferidos.


 \needspace{5\baselineskip}\section{Submissão}

Nada revela mais sobre a maturidade de uma pesquisa do que apertar o botão ``submit''. O processo costuma ser:

\begin{enumerate}
\item Definição do artigo;
\item Início da escrita ou de uma versão inicial de trabalho;
  \item Escolha  do periódico ou conferência, analisando
        fator de impacto, escopo temático e tempo de resposta;
    \item Determinação de como deve ser um artigo para o meio escolhido;
    \item Escrita do artigo de acordo com o estilo do meio;
  \item Checklist de integridade: ORCID, conflitos de interesse, aprovações do comitê de ética, declaração de dados abertos;
  \item Ajuste fino de formatação, cuidando do template, formatos de referência, figuras em alta resolução;
  \item Submissão; 
  \item Acompanhamento da revisão; \label{passo:acompanhar}
  \item Se irritar com as respostas, mesmo quando positivas em geral;
  \item Realização os acertos necessários; 
  \item Envio a versão (supostamente) final;
  \item Espera da publicação, e
  \item Atualização do Lattes e divulgação entre os pares.
\end{enumerate}

\gxatencao{Nunca submeta um mesmo artigo para dois lugares simultaneamente.}


 \needspace{5\baselineskip}
 \section{Revisão por pares}

Após uma primeira vista do Editor, que pode devolver, por achar que o artigo não cabe na revista, ou recusa imediata, normalmente a revisão é feita por dois ou mais revisores, no formato ``duplamente cego''.  Isso significa que os revisores não sabem quem são os autores e vice-versa. 

O processo, apesar de buscar garantir a qualidade das publicações, é muito cansativo para os editores e revisores e traumático para os revisados. 

É comum que, ao ler uma revisão, os autores encontrem erros, falta de respeito com o trabalho e pedidos absurdos, como refazer todos os experimentos, considerar uma teoria completamente diferente ou genérica demais. 
Já recebi revisões que simplesmente diziam que um bom resultado devia ser mentira e recusavam o artigo com base nessa hipótese, e revisões de outro artigo, como se fossem o meu. Em congressos, onde o processo é massivo e rápido, os erros são mais comuns.

As respostas normalmente se enquadram em: aceitação, aceitação com revisões menores (\textit{minor revision}), o que é praticamente a aceitação, aceitação com revisões maiores (\textit{major revision}), o que pode variar entre uma recusa educada ou uma aceitação com grandes exigências, o que até mesmo pode levar a desistência dos autores de seguir o processo com aquela revista, recusa pelo resultado da revisão, recusa pelo editor (por não ser do tema da revista ou não atingir parâmetros mínimos, até de formatação), ou ainda devolução pelo editor por não encontrar revisores ou achar que o artigo não é do tema da revista.

Mesmo aceitos, muitos artigos passam ainda por um processo de publicação que inclui revisões simples, provas de impressão ou outros pedidos de revisão.

Ao receber o parecer, e este não foi uma recusa, siga esta sequência de sobrevivência:

\begin{enumerate}
  \item Leia tudo uma primeira vez.  
  \item Deixe a raiva decantar por 24 h.  
  \item Construa uma tabela \textit{Pedido do Revisor, Proposta de Resposta}. Considere que alguns pedidos podem não ser realizáveis, e possivelmente representam na prática uma recusa. 
  \item Chame os co-autores para discutir o que será feito, que inclui:
  \begin{itemize}
      \item Fazer as revisões, possivelmente todas,  para a mesma revista, se isso é possível;
      \item Fazer as revisões e enviar para outra revista;
      \item Fazer algumas revisões, e
      \item Variações sobre o tema...
  \end{itemize}
\end{enumerate}

É importante enviar, junto com o novo artigo, uma carta explicando tudo que foi feito, o que foi atendido e contrapondo o que decidiram não fazer. 
A tabela ajuda a escrever a carta. 
Faça isso mesmo que ela não seja obrigatória. 
Essa carta deve ser muito detalhada, apontando e repetindo, se possível, as modificações no texto. 
Por exemplo, você não deve só dizer que trocou a figura 1, mas também mostrar a anterior, a nova, e escrever como as modificações atendem o pedido do revisor.

\gxatencao{Se seu artigo foi recursado, aproveite os comentários e faça uma versão para uma nova revista ou congresso.}


 \needspace{5\baselineskip}\section{A lenda do Revisor 2}

No folclore acadêmico, sempre há um Revisor 2 pronto a criticar seu trabalho de forma errônea ou cruel. Entre os poderes dessa super-vilão estão:
\begin{description}
  \item A Onisciência, pois conhece todos os \emph{state of the art} publicados na semana passada;
  \item A Limitação, pois desconhece outras áreas e tem dificuldade de entender abordagens interdisciplinares;
  \item A Telepatia, pois detecta suposições não escritas e quer ver todas testadas, e 
  \item A Contradição, pois pede mais informações, mas sugere reduzir o artigo.
\end{description}

Para lidar com isso:
\begin{enumerate}
  \item \textbf{Não personalize}, o ataque ao trabalho não é um        ataque à pessoa;
  \item \textbf{Busque a causa raiz} do ataque e como pode ser resolvido;
  \item \textbf{Converta sugestões em ação}, se a amostra é pequena,    inclua análise de poder estatístico; se faltam referências, cite-as;
  \item \textbf{Negocie com dados}, quando discordar,
        explique por que determinada modificação comprometeria o método ou extrapolaria os limites do estudo.
\end{enumerate}

Submissão e revisão compõem um ciclo virtuoso: 
construímos, testamos, corrigimos, reconstruímos.  
E, à medida que enfrentamos cada novo Revisor 2, fica claro que
publicar é menos sobre ser perfeito e mais sobre ser 
\emph{iterativamente melhor}. Muitas vezes, um artigo recusado se torna muito melhor quando as sugestões são seguidas e fazem sucesso em outro lugar.

