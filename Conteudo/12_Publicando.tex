\chapter{Publicando}

\chapter{Resultados}

Só existe uma maneira verdadeiramente honesta de avaliar um trabalho científico: submetê-lo a revisão de seus pares. Por isso existe uma banca de mestrado e doutorado. Por isso cada vez mais é importante publicar seus resultados.

Na forma atual que alunos, professores e programas de pós-graduação são avaliados é impossível imaginar uma tese onde não houve uma publicação.
A política correta de um orientador deve ser não permitir a defesa de uma tese que não tenha nenhum artigo publicado. Para isso dou dois motivos: se nenhum trabalho foi apresentado para a publicação, então o aluno não demonstrou interesse, se nenhum trabalho foi aceito, a tese não demonstrou capacidade.

Publicar é responsabilidade do aluno. Cabe ao orientador auxiliá-lo nessa tarefa. Claro que, dependendo da capacidade do orientador na área, ele pode ser a força motriz do artigo. Porém, é importante que o aluno tenha a experiência de conduzir a parte principal do trabalho de publicação.

Uma tese de doutorado tem uma obrigação ainda maior: publicar artigos em revista.

Publique sempre. Antes de acabar a tese, depois de acabar a tese. A única maneira de seu trabalho ficar conhecido e você ser reconhecido é por meio de publicações. Aceite ligeiros atrasos em sua tese (que não interfiram com seu prazo) se for para publicar. Publicar dará pontos em concursos públicos para professor e tornará você conhecido na comunidade.

Ao publicar, não esqueça que os autores são, pelo menos, você e seu orientador. Geralmente o aluno vem em primeiro lugar, mas algumas vezes, principalmente quando a ideia principal vem do orientador, o nome dele vem em primeiro. Publicar sem o nome do orientador é um dos maiores pecados que um aluno pode fazer contra a relação aluno/orientador na área da Computação.

A questão da publicação está se tornando cada vez mais séria no Brasil. Tanto a CAPES quanto as universidades estão avaliando seus pesquisadores principalmente em função da quantidade e qualidade das publicações.

\section{A Pressão para Publicar Frequentemente}

Devido às metodologias de avaliação a que estão submetidos os diversos programas de pós-graduação, publicar se tornou uma atividade imperativa ao longo de dissertações de mestrado e teses de doutorado.

Normalmente são feitas as seguintes avaliações que consideram as publicações:

\begin{itemize}
  \item Os professores são avaliados dentro de seus programas de pós-graduação, para poder orientar
  \item Os professores são avaliados para promoções
  \item Os professores são avaliados quando fazem pedidos de bolsa ou projetos
  \item Os programas são avaliados pela Capes e pelo CNPq
  \item As universidades são avaliadas pelo MEC, por organismos nacionais e internacionais 
\end{itemize}

\section{A primeira regra da publicação é}

\gxatencao{Escrever sempre.}

Tudo que você fizer deve levar em conta a possibilidade de uma publicação. Se não permite uma publicação é porque provavelmente não há contribuição.

\section{A segunda regra é}

\gxatencao{Sempre envolva seu orientador.}

Na Computação é praxe que os artigos sejam feitos com a colaboração direta dos orientadores e alunos. Em outras áreas isso pode não ser verdade. A norma é só retirar o nome se o orientador pedir e sempre apresentar o artigo ao orientador antes de submetê-lo a um congresso ou revista. Um orientador espera que você trabalhe com ele. Ele também evitará que você cometa erros grosseiros por inexperiência. 

\section{A terceira regra é}

\gxatencao{Relacione como autores todos os envolvidos.}

É melhor errar por incluir algum autor que não merecia estar citado do que excluir um que merecia. Não se esqueça de enviar o artigo a todos os autores, pedir sua colaboração e aceitação. Cabe a quem não se interessar solicitar a retirada do seu nome do artigo. 

Sempre se pergunte: esse texto seria produzido, da forma como está, caso a pessoa específica não tivesse dado sua colaboração? 

\section{Ordem de Autores}

A questão da ordem de autores é normalmente resolvida colocando em ordem decrescente de manipulação do texto. Normalmente, o autor principal é aquele que contribuiu mais para as partes mais importantes do texto. 

\section{Escolhendo onde publicar}

Para publicar, selecione congressos e revistas conceituadas e de impacto. 

Muitas universidades têm acesso a sistemas que permitem verificar o impacto das publicações. Além disso, quase todas as áreas já conseguiram organizar seu ``Qualis'', uma lista mantida pela CAPES classificada das publicações que tem como objetivo indicar sua qualidade. 

O primeiro alvo de suas publicações devem ser os mesmos congressos e revistas que você usa na sua bibliografia.

Tenha muito cuidado com o que se costumou chamar de publicações ou editoras predatórias, que aceitam qualquer texto, muitas vezes em troca de uma taxa. Muitas foram criadas para dar um fórum a trabalhos de má qualidade, e precisam de trabalhos de boa qualidade para serem validadas. O problema é que, quando detectadas, tudo que é publicado lá é considerado lixo. Seu orientador geralmente terá mais facilidade de reconhecer algo desse tipo, mas existem listas na internet denunciando publicações desse tipo.

\section{Plágio e Citações}

Basicamente, plagiar significa apresentar como seu trabalho que foi feito e já publicado por outro. No mundo acadêmico, o plágio é considerado uma desonestidade séria e é punido de várias formas, formais e informais, como a exclusão de um curso, a reprovação de um trabalho ou em uma cadeira, a demissão de professores e até mesmo, em alguns casos, sendo levado à justiça comum.

Isso significa que todo texto para o qual assumimos a autoria deve ser original, sob o risco de incorrer em plágio. Obviamente, não é possível fazer trabalho científico sem se utilizar de ideias e textos de outros autores como ponto de partida e apoio, logo existem regras claras de como realizar citações, isto é, como descrever o trabalho de outro de forma que fique clara a atribuição de autoria.

No Brasil existe uma norma de citação mantida pela ABNT e muitas universidades mantêm versões próprias, inspiradas na ABNT. Nessas normas se descrevem, de forma bastante detalhada, as várias maneiras de se declarar uma citação. Nem sempre, porém, fica claro o que é uma citação.

Existem duas formas de citação: a citação direta e a citação indireta. 

Na citação direta copiamos diretamente o texto do autor e, por causa disso, devemos marcar de forma clara que estamos fazendo essa cópia. Segundo a norma ABNT isso é feito pelo uso de aspas, quando a citação tem até 3 linhas, ou usando um parágrafo com recuo de 4 cm da margem esquerda (ABNT, 2001).

\begin{quote}
``citações indiretas (ou livres) são a reprodução de algumas idéias, sem que haja transcrição das palavras do autor consultado. Apesar de ser livre, deve ser fiel ao sentido do texto original. Não necessita de aspas.'' (ABNT, 2001)
\end{quote}

Nos dois parágrafos acima fizemos uma citação indireta ao descrever a direta e uma citação direta ao descrever uma indireta. 

\section{Como trabalhar um artigo em grupo}

Apesar de facilitar o trabalho, escrever um artigo em grupo, principalmente se feito remotamente, cada um em um local diferente, não é uma tarefa fácil. O trabalho de um pode destruir o trabalho de outro, estilos podem ficar misturados, etc.

Algumas recomendações:

\begin{itemize}
  \item É possível fazer grande parte do trabalho textual em um editor on-line. Não é minha opção preferida, mas mantém tudo sincronizado. As opções são óbvias: Google Docs e a versão nuvem do Microsoft Word. 
  \item Overleaf é uma boa opção para o mundo LaTeX.
  \item A versão final deve provavelmente ser feita em Word (ou LaTeX). Para isso use o DropBox, OneDrive, Google Drive, compartilhando pastas, ou serviço semelhante. 
  \item Outra solução é usar o GitHub.
  \item Crie um diretório específico para o trabalho do artigo e compartilhe com todos.
  \item O nome do arquivo deve ser algo do tipo \texttt{Nome do artigo – parte – versão – autor que fez a versão.docx}.\\
  Exemplo: \texttt{theboss texto v10 xexeo.docx}
  \item Se usar o GitHub, ou outro sistema de controle de versões, esse problema acaba.
\end{itemize}

Mantenha a última versão do artigo na raiz desse diretório, junto com a template original da revista ou congresso e o call for papers.

Mantenha diretórios separados que indicam o que está nele, como em:

\begin{itemize}
  \item \textbf{Versões antigas}
  \item \textbf{Subsídios} (contendo cópias de todas as referências usadas e até mesmo algumas não usadas, como referência de escrita)
  \item \textbf{Dados} (todos os dados usados para escrever o artigo, na medida do razoável pelo consumo de espaço)
  \item \textbf{Programas} (todos os programas usados para processar os dados)
  \item \textbf{Resultados} (todos os resultados obtidos)
  \item \textbf{Software} (links ou software usado no processamento, na medida do razoável para utilização)
\end{itemize}

\begin{itemize}
  \item Mantenha a revisão ligada no Word e no Overleaf.
  \item Sempre que grandes mudanças forem feitas, faça uma nova versão do artigo, criando uma versão e passando as velhas para o diretório de versões antigas.
  \item Possivelmente quebre o artigo em partes.
  \item Mantenha o formato da conferência o mais cedo possível, para ter ideia de tamanho.
  \item Se comunique via meios eletrônicos para registrar as comunicações. 
  \item Em certo ponto, congele o crescimento do artigo, para se preparar apenas para revisões e correções. 
  \item Ao congelar, peça para todos os autores fazerem sua última revisão. 
  \item Tente dar ao menos 24 horas para isso.
  \item Marque uma reunião final para fechar o arquivo e para que todos possam submeter conjuntamente ou concordar que um será responsável pela submissão.
\end{itemize}

Para artigos com poucas citações, pode ser mais fácil usar o \textit{Citation Machine} (\url{http://www.citationmachine.net/}) ou o próprio Word para gerar suas citações ou fazê-las manualmente, facilitando também o trabalho multi-autor.

Não se esqueça que o LaTeX usa o \texttt{bibTeX}.

Não se esqueça que existem softwares muito poderosos de controle de referência que serão essenciais à sua tese.

