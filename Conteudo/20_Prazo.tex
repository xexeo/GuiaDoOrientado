\chapter{O Prazo}

Como todo projeto, uma dissertação ou tese tem que ter um fim.

A COPPE limite o prazo de uma \textbf{dissertação em 3 anos}, mais meio ano de uma possível extensão que, pelo regulamento, não será dada facilmente.

O exame de \textbf{qualificação de mestrado} tem prazo de \textbf{dois anos sem extensão possível}.

O limite de uma \textbf{tese é de 5 anos}, com possível \textbf{extensão} (que novamente provavelmente não será dada) \textbf{de 1 ano}, com \textbf{o exame de qualificação tendo prazo de 3 anos, sem extensão possível}.

Você deve tentar atingir não os prazos máximos da COPPE, mas sim os prazos das bolas: mestrado 2 anos, doutorado 4 anos.

Sabemos que esses prazos são pequenos e eu sugiro que no máximo se chegue a 2,5 anos e 4,5 anos.

O que acontece se você usa o prazo máximo?

Vamos esquecer, momentaneamente, que você pode PERDER O PRAZO, o que é perder o trabalho de anos. Quais os outros problemas?

Primeiro, o texto sai muito pior do que devia, os experimentos são terminados de forma açodada. A banca vai reclamar muito e provavelmente reprovar com restrições e te dar um dever de casa.

Segundo, a capacidade do orientador ajudar em algo nesses casos (fim do seu prazo) é muito reduzida. Por quê? Porque seu orientador tem outras coisas para fazer no trabalho e uma vida pessoal\footnote{Aconteceu comigo, tive uma doença grave junto com o prazo final de muito alunos, incluindo 6 meses de licença, 50  dias de hospital e uma cirurgia de coração aberto. Conseguimos resolver o problema de todos, mas não foi uma boa experiência para ninguém. } , que inclui outros orientados provavelmente também fazendo a mesma coisa. Isso significa que ele, em especial eu, não vai poder realmente orientar o seu final de tese, o que é muito ruim, mas é o que acontece na prática.

Não é razoável você esperar que depois de ficar 5(3) anos fazendo seu trabalho sem muito interesse, no último instante queira uma dedicação emergencial do seu orientador. Ele estava lá por muito tempo. Pode ser que, algumas vezes, ele consiga essa dedicação, mas também pode ser que não.

Ou seja, o seu risco, como orientado, aluno e candidato ao título cresce quando o prazo está chegando. Cuidado!
