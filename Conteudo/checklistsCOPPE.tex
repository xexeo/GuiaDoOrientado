\chapter{Checklists Específicos para a COPPE}


\section{Checklist de Pré Requisitos}

\begin{itemize}[label=\(\square\),leftmargin=2em,nosep]
    \item  Verificar sua data de matrícula no SIGA e calcular o prazo máximo para defesa.
    \item  Conferir se já realizou e teve sua candidatura homologada no Seminário de Qualificação do Mestrado.
\end{itemize}


\section{Checklist Pré-Defesa}

\begin{itemize}[label=\(\square\),leftmargin=2em,nosep]
    \item  Discutir a banca com o orientador pelo menos 60 dias antes da defesa.
    \item  Providenciar o PDF com o currículo Lattes do membro externo da banca, se brasileiro, ou outro currículo, se estrangeiro, e enviar ao orientador.
    \item  Concordar com a defesa híbrida ou remota por meio de documento específico.
    \item  Pedir ao orientador para fazer o pedido de banca com pelo menos 60 dias de antecedência.
    \item  Garantir que o pedido de banca foi feito com pelo menos 45 dias de antecedência da primeira data planejada.
    \item  Verificar no SEI a aprovação da banca.
    \item  Definir a data final da defesa com a banca e orientador.
    \item  Reservar a sala presencial ou criar o link da sala virtual.
    \item  Depositar a dissertação (PDF) no \verb|ctrl-pesc| até 15 dias antes da defesa.
    \item  Enviar o PDF da dissertação a todos os membros da banca até 15 dias antes da defesa.
    \item  Levantar, preparar, assinar e enviar ao registro e aos orientadores os documentos necessários para pedir a ata, constante da informação de pré-defesa da Coppe.
    \item  Garantir que o orientador solicitou a ata à secretaria (digital ou física).
    \item  Confirmar que os documentos obrigatórios foram entregues ao registro (assinados).
    \item  Confirmar que o orientador divulgou o anúncio formal da defesa no site do PESC (via secretaria).
    
\end{itemize}

\section{Checklist no Dia da Defesa}

\begin{itemize}[label=\(\square\),leftmargin=2em,nosep]
    \item  Chegar com antecedência à sala física ou testar acesso à sala virtual.
    \item  Levar computador próprio com a apresentação pronta e funcional offline.
    \item  Ter pen-drive de backup com os slides.
    \item  Levar garrafa de água para auxiliar caso haja nervosismo.
    \item  Conferir funcionamento de câmera, microfone e fones de ouvido.
    \item  Ter plano B para internet (tethering de celular, etc).
    \item  Conferir se a ata está em posse do presidente da banca.
    \item  Preparar local alternativo caso haja queda de luz/internet.
    \item  Conferir que todos os membros da banca estão com o link correto.
    \item  Ter bloco de anotações (ou editor de texto) para registrar as falas da banca.
\end{itemize}


\section{Checklist Pós-Defesa}

\begin{itemize}[label=\(\square\),leftmargin=2em,nosep]
    \item  Conferir se a ata foi assinada e entregue pelo orientador ao registro.
    \item  Se aprovado com exigências: seguir instruções da folha de modificações.
    \item Se aprovado com exigências: Entrar em contato com fiscal e orientador para confirmar se as exigências foram cumpridas.
    \item  Preparar a versão final da dissertação.
    \item  Depositar a versão final no sistema da Coppe (registro acadêmico), respeitando o prazo.
    \item  Depositar a versão final no site do PESC.
    \item  Enviar a versão final (PDF) para todos os membros da banca.
    \item  Solicitar emissão de diploma após confirmação dos depósitos.
\end{itemize}


