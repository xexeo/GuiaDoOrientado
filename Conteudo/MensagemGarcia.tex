\chapter{Mensagem a Garcia}

Um texto final para motivar os orientados.

Mensagem a Garcia Elbert Hubbard – fevereiro de 1899

Em todo este caso cubano, um homem se destaca no horizonte de minha memória. Quando irrompeu a guerra entre a Espanha e os Estados Unidos, o que importava aos americanos era comunicar-se, rapidamente, com o chefe dos revoltosos – chamado Garcia - que se encontrava em uma fortaleza desconhecida, no interior do sertão cubano. Era impossível um entendimento com ele pelo correio ou pelo telégrafo. No entanto, o Presidente precisava de sua colaboração, e isso o quanto antes. Que fazer? Alguém lembrou: ``Há um homem chamado Rowan... e se alguma pessoa é capaz de encontrar Garcia, esta pessoa é Rowan''.

Rowan foi trazido à presença do Presidente, que lhe confiou uma carta com a incumbência de entregá-la a Garcia. Não vêm ao caso narrar aqui como esse homem tomou a carta, guardou-a num invólucro impermeável, amarrou a ao peito e, após quatro dias, saltou de um pequeno barco, alta noite, nas costas de Cuba; ou como se embrenhou no sertão para, depois de três semanas, surgir do outro lado da ilha, tendo atravessado a pé um país hostil, e entregue a carta a Garcia. O ponto que desejo frisar é este: Mac Kinley deu a Rowan uma carta destinada a Garcia; Rowan tomou-a e nem sequer perguntou: ``Onde é que ele está?''.

Eis aí um homem cujo busto merecia ser fundido em bronze e sua estátua colocada em cada escola. Não é só de sabedoria que a juventude precisa... Nem de instruções sobre isto ou aquilo. Precisa, sim, de um endurecimento das vértebras para poder mostrar-se altiva no exercício de um cargo; para atuar com diligência; para dar conta do recado; para, em suma, levar uma mensagem a Garcia. O General Garcia já não é deste mundo, mas há outros Garcias. A nenhum homem que se tenha empenhado em levar adiante uma tarefa em que a ajuda de muitos se torne precisa tem sido poupados momentos de verdadeiro desespero ante a passividade de grande número de pessoas ante a inabilidade ou falta de disposição de concentrar a mente numa determinada tarefa... e fazê-la. A regra geral é: assistência regular, desatenção tola, indiferença irritante e trabalho malfeito.

Ninguém pode ser verdadeiramente bem-sucedido, exceto se lançar mão de todos os meios ao seu alcance, para obrigar outras pessoas a ajudá-lo, a não ser que Deus Onipotente, na sua grande misericórdia, faça um milagre enviando-lhe, como auxiliar, um anjo de luz. Leitor amigo, tu mesmo podes tirar a prova. Está sentado no teu escritório, rodeado de meia dúzia de empregados. Pois bem, chama um deles e pede-lhe: ``Queria ter a bondade de consultar a enciclopédia e de fazer a descrição resumida da vida de Corrégio''.

Dar-se-á o caso de o empregado dizer, calmamente: –  ``Sim, senhor'' e executar o que lhe pediste? Nada disso! Olhar-te-á admirado para fazer uma ou algumas das seguintes perguntas:

---  Quem é Corrégio?

---  Que enciclopédia?

---  Onde está a enciclopédia?

---  Fui contratado para fazer isso?

--- E se Carlos o fizesse?

---  Esse sujeito já morreu?

--–  Precisa disso com urgência?

--–  Não seria melhor eu trazer o livro para o Senhor procurar?

–-- Para que quer saber isso?

Eu aposto dez contra um que, depois de haveres respondido a tais perguntas e explicado a maneira de procurar os dados pedidos, e a razão por que deles precisas, teu empregado irá pedir a um companheiro que o ajude a encontrar Corrégio e depois voltará para te dizer que tal homem nunca existiu. Evidentemente pode ser que eu perca a aposta, mas, seguindo uma regra geral, jogo na certa. Ora, se fores prudente, não te darás ao trabalho de explicar ao teu "ajudante" que Corrégio se escreve com ``C'' e não com ``K'', mas limitar-te-á a dizer calmamente, esboçando o melhor sorriso: ``Não faz mal... não se incomode''. É essa dificuldade de atuar independentemente, essa fraqueza de vontade, essa falta de disposição de, solicitamente, se por em campo e agir, é isso o que impede o avanço da humanidade, fazendo-o recuar para um futuro bastante remoto. Se os homens não tomam a iniciativa de agir em seu próprio proveito, que farão se o resultado de seu esforço resultar em benefício de todos? Por enquanto parece que os homens ainda precisam ser dirigidos.

O que mantém muitos empregados no seu posto e o faz trabalhar é o medo de, se não o fizer, ser despedido ou transferido no fim do mês. Anuncia-se precisar de um taquígrafo e nove entre dez candidatos à vaga não saberão ortografar nem pontuar, e –  o que é pior – pensa não ser necessário sabê-lo.

``Olhe aquele funcionário'' -–  dizia o chefe de uma grande fábrica. É um excelente funcionário. Contudo, se eu lhe perguntasse por que seu trabalho é necessário ou por que é feito dessa maneira e não de outra, ele seria incapaz de me responder. Nunca deve ter pensado nisso. Faz apenas aquilo que lhe ensinaram, há mais de 3 anos, e nem um pouco a mais".

``Será possível confiar-se a tal homem uma carta para entregá-la a Garcia?''.

Conheço um homem de aptidões realmente brilhantes, mas sem a fibra necessária para dirigir um negócio próprio e que ainda se torna completamente nulo para qualquer outra pessoa devido à suspeita que constantemente abriga de que seu patrão o esteja oprimindo ou tencione oprimi-lo. Sem poder mandar, não tolera que alguém o mande. Se lhe fosse confiada a mensagem a Garcia retrucaria, provavelmente:

--–  Leve-a você mesmo!.

Hoje esse homem perambula errante, pelas ruas em busca de trabalho, em estado quase de miséria. No entanto, ninguém se aventura a dar-lhe trabalho porque é uma personificação do descontentamento e do espírito da discórdia. Não aceitando qualquer conselho ou advertência, a única coisa capaz de nele produzir algum efeito seria um bom pontapé dado com a ponta de uma bota 44, sola grossa e bico largo.

Pautemos nossa conduta por aqueles homens, dirigente ou dirigida, que realmente se esforçam por realizar o seu trabalho. Aqueles cujos cabelos ficam mais cedo envelhecidos na incessante luta que estão desempenhando contra a indiferença e a ingratidão, justamente daqueles que, sem o seu espírito empreendedor, andariam famintos e sem lar.

Estarei pintando o quadro com cores por demais escuras?

Não há excelência na nobreza de si mesmo; farrapos não servem de recomendação. Nem todos os ricos são gananciosos e tiranos, da mesma forma que nem todos os pobres são virtuosos. Todas as minhas simpatias pertencem ao homem que trabalha, fazendo o que deve ser feito, melhorando o que pode ser melhorado, ajudando sem exigir ajuda. É o homem que, ao lhe ser confiada uma carta para Garcia, toma a missiva e, sem a intenção de jogá-la na primeira sarjeta, entrega-a ao destinatário. Esse homem nunca ficará "encostado", nem pedirá que lhe façam favores.

A civilização busca ansiosamente, insistentemente, homens nessa condição. Tudo que tal homem pedir, se lhe há de conceder. Precisa-se dele em cada vila, em cada lugarejo, em cada escritório, em cada oficina, em cada loja, fábrica ou venda. O grito do mundo inteiro praticamente se resume nisso:

\gxatencao{PRECISA-SE –  E PRECISA-SE COM URGÊNCIA –  DE UM HOMEM CAPAZ DE LEVAR UMA MENSAGEM A GARCIA.}
