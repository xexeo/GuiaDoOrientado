\chapter{Hábitos e Práticas}


\section{Ler}


A leitura é a única forma do candidato a um título alcançar a maturidade necessária para defendê-lo. Muitos alunos têm uma boa ideia e acreditam que realizá-la e descrevê-la caracteriza uma tese. Isso não é verdade. 


\gxatencao{O aluno deve ler.}


Uma tese tem que ser colocada no contexto científico atual e comparada com os trabalhos já realizados sobre o assunto ou sobre temas similares ou análogos. Deve ficar claro, na tese, qual a colaboração que o trabalho traz a ciência. Obviamente, só é possível fazer isso se o aluno tem um conhecimento da área, que deve ser, na maior parte das vezes, até mesmo superior ao conhecimento do orientador.


Quando o aluno não lê o suficiente isso fica muito claro para o orientador. Há uma falta de capacidade de argumentação, uma falta de base teórica para o trabalho. Um dos principais sinais de maturidade que um orientador percebe é a capacidade de argumentação baseada em evidências científicas.


É importante notar que não basta ler, mas é necessário ler publicações atualizadas (além dos textos clássicos da área).


Se você achar que está lendo pouco, ou se seu orientador reclamar disso, aqui estão algumas dicas:

\begin{outline}	
\1	Levante uma lista de congressos e revistas relativas à área de sua tese.
\2	Faça um mix com o máximo possível de publicações da área específica com publicações importantes da área mais geral
\1	Obtenha os últimos cinco anos desses congressos e revistas.
\1	Liste todos os artigos disponíveis que possam servir para sua tese
\1	Obtenha esses artigos
\1	Leia os resumos e os organize de alguma forma, priorizando a leitura
\2	Procure tutoriais e reviews para o início da leitura
\2	Leia alguns artigos clássicos citados nos artigos obtidos
\2 Leia os artigos específicos, com foco nos mais atuais
\1	Mantenha o controle dessa lista e faça o acompanhamento com o orientador.
\end{outline}

A maioria dos textos de metodologia científica recomenda o fichamento dos artigos lidos. Essa prática é importante, porém não é mais necessário usar fichas. Você pode usar um banco de dados, um sistema de referência ou até mesmo um ou mais arquivos de documento, como arquivos Word. Até mesmo “Post-it” podem gerar um bom sistema de fichamento. 


\subsection{O Aluno que só lê português (e não conta ao orientador)}


Dificilmente você será aceito no mestrado se sua capacitação em inglês não permite uma leitura em ritmo razoável, porém isso pode acontecer.


Nesse caso, deixe bem claro ao orientador sua dificuldade. Esconder qualquer dificuldade de leitura ou compreensão, seja ela de inglês ou de alguma matéria específica, fará com que seu orientador avalie sua dificuldade como falta de dedicação.


O resultado é que, em vez de o orientador ajudá-lo nessa dificuldade, ele tornará as coisas cada vez mais difíceis.


Por sinal, se esse for seu caso, entre imediatamente em um curso de inglês. Qualquer melhoria, junto com a leitura de textos da área, implicará em um rendimento maior do seu trabalho. Existem cursos gratuitos ou muito baratos na universidade.


\section{Escrever}


É quase impossível seguir uma carreira acadêmica em qualquer área de sem escrever bem, pelo menos em português. 


Se você acha que escreve mal, ou se os outros acham que escreve mal, tente corrigir o mais rápido possível. Faça cursos e se esforce. Muitos alunos simplesmente acham “normal” um profissional de área técnica escrever mal. Isso demonstra uma falta de compreensão das necessidades do mundo real: apresentar e defender, de forma clara, suas ideias. 


Como diria o Chacrinha\footnote{É provável que alguns dos leitores mais novos não tenham conhecido o “Chacrinha”. Ele foi por muitos anos um apresentador de programa de auditório de muito sucesso, usando uma fantasia e utilizando bordões engraçados. Certamente foi uma das figuras mais conhecidas na TV brasileira do século XX.}: 


\gxatencao{Quem não se comunica, se trumbica}


Uma das principais indicações da educação de uma pessoa é sua capacidade de se expressar em sua língua mãe.


Ao estudante universitário, essa característica é muito desejada. Ao aluno de mestrado e doutorado, é indispensável.


Contribuição de um aluno que aprendeu a escrever.

\quote{Só consegue escrever quem consegue dizer o que pensa e quer.  Fale, diga o que pensa, conte para seu orientador suas ideias e busque clareza ao dizer.

Escreva simples.  Uma tese é, antes de qualquer coisa, uma coleção de folhas de papel com um monte de letrinhas em cima.  

Ler é, depois de falar, uma das melhores ajudas para quem quer aprender a escrever.  Ler tudo, de jornal a bula de remédio, passando por contos, poesias, história em quadrinhos, artigos e, naturalmente, uma tese ou outra de vez em quando.

Depois fica-se assim, querendo escrever em qualquer lugar, em qualquer oportunidade, em qualquer Wiki que se abra.

Desejo boa sorte a nós todos.

\textit{Bebeto}}


