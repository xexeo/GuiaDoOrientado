% !TEX encoding = UTF-8
% Checklist — Pronto para Defender
\chapter{Checklist para Defesa}

\begin{itemize}[label=\(\square\),leftmargin=*]

  % Conteúdo Acadêmico
  \item \textbf{Conteúdo Acadêmico}
    \begin{itemize}[label=\(\square\),leftmargin=2em,nosep]
      \item Pergunta de pesquisa claramente formulada e alinhada ao problema (``Por quê?'' e ``Para quem?'').
      \item Contribuição original explicitada e mapeada à lacuna na literatura.
      \item Revisão bibliográfica completa, atualizada e criticamente discutida.
      \item Metodologia descrita em detalhe suficiente para replicação (dados, instrumentos, scripts).
      \item Resultados analisados com estatísticas/figuras adequadas e relação direta com hipóteses.
      \item Discussão que articula resultados, teoria e implicações práticas.
      \item Limitações reconhecidas e agenda de pesquisas futuras delineada.
    \end{itemize}

  % Produção Científica
  \item \textbf{Produção Científica}
    \begin{itemize}[label=\(\square\),leftmargin=2em,nosep]
      \item Artigo(s) submetido(s)/aceito(s) conforme requisito do programa (anexar comprovantes).
      \item Dados e código depositados em repositório aberto ou institucional (DOI gerado).
    \end{itemize}

  % Manuscrito
  \item \textbf{Manuscrito}
    \begin{itemize}[label=\(\square\),leftmargin=2em,nosep]
      \item Estrutura obedece às normas da pós‑graduação (sumário, capítulos, anexos).
      \item Formatação, citações e bibliografia em conformidade (BibLaTeX com campos \texttt{date} e \texttt{urldate}).
      \item Revisão linguística concluída; estilo uniforme (voz, tempo verbal, terminologia).
      \item Relatório de similaridade  (se software antiplágio é obrigatório).
    \end{itemize}

  % Aprovação Interna
  \item \textbf{Aprovação Interna}
    \begin{itemize}[label=\(\square\),leftmargin=2em,nosep]
      \item Orientador revisou a versão final e deu \emph{ok} por escrito.
      \item (se existe) Co‑orientador ou comissão de acompanhamento emitiram parecer favorável.
      \item Pareceres anteriores (qualificação, seminários) foram plenamente atendidos.
    \end{itemize}

  % Documentação Administrativa
  \item \textbf{Documentação Administrativa}
    \begin{itemize}[label=\(\square\),leftmargin=2em,nosep]
      \item Formulários de depósito da tese/dissertação preenchidos e assinados.
      \item Banca examinadora homologada pel órgão responsável.
      \item Versão ``para banca'' enviada aos membros no prazo institucional.
    \end{itemize}

  % Preparação da Defesa
  \item \textbf{Preparação da Defesa}
    \begin{itemize}[label=\(\square\),leftmargin=2em,nosep]
      \item Data, horário e local (ou link remoto) confirmados com todos os membros.
    \item Apresentação de slides concluída com a duração testada, com slides numerados.
      \item Ensaio realizado com colegas; perguntas recorrentes catalogadas e roteirizadas.
      \item Arquivos de backup em múltiplos formatos (PDF, PPTX, vídeo) e cópia na nuvem/pen‑drive.
      \item Material de apoio impresso (folha‑resumo, \emph{handout}) preparado, se exigido.
    \end{itemize}

  % Pós‑Defesa
  \item \textbf{Pós‑Defesa (pré‑agendado)}
    \begin{itemize}[label=\(\square\),leftmargin=2em,nosep]
      \item Templates de errata e ficha catalográfica já configurados.
      \item Plano de submissão da versão final ao repositório institucional no prazo hábil.
      \item  Atualizar seu currículo Lattes com o título da dissertação e nome dos orientadores.
    \end{itemize}

\end{itemize}
