% Comandos criados

% star - cannot contain \par (see https://tex.stackexchange.com/questions/1050/whats-the-difference-between-newcommand-and-newcommand)
\newcommand*{\gxdefine}[1]{\textbf{#1}\index{#1}}
\newcommand*{\gxdefineplus}[2][]{\textbf{#2 #1}\index{#2@\textit{#2}\ifx#1\empty\else!\textit{#1}\fi}}

\newcommand{\Excel}{Excel\texttrademark}
\newcommand{\ExcelScale}{0.4}
\newcommand{\furl}[1]{\footnote{\url{}}}

% can´t remember what it does
\makeatletter
\DeclareRobustCommand*{\escapeus}[1]{%
    \begingroup\@activeus\scantokens{#1\endinput}\endgroup}
\begingroup\lccode`\~=`\_\relax
    \lowercase{\endgroup\def\@activeus{\catcode`\_=\active \let~\_}}
\makeatother

%% DESENHA CUBOS!
\newcommand{\drawbox}[5]{
    \pgfmathsetmacro \angle {30}
    \pgfmathsetmacro \xd {{2/3*cos(\angle)*#5}}
    \pgfmathsetmacro \yd {{2/3*sin(\angle)*#5}}
    \pgfmathsetmacro \x {{#1-#5+(#2-#5)*(\xd)*#5}}
    \pgfmathsetmacro \y {{#3-#5+(#2-#5)*(\yd)*#5}}

    \draw[fill=#4] (\x,\y) -- (\x+#5,\y) -- (\x+#5,\y+#5) -- (\x,\y+#5) -- cycle;

    \draw[fill=#4] (\x,\y+#5) -- (\x+\xd,\y+#5+\yd) -- (\x+#5+\xd,\y+#5+\yd) -- (\x+#5,\y+#5) -- cycle;

    \draw[fill=#4] (\x+#5,\y+#5) -- (\x+#5+\xd,\y+#5+\yd) -- (\x+#5+\xd,\y+\yd) -- (\x+#5,\y) -- cycle;
}

% desenha linhas malucas
\newcommand\irregularline[2]{%

%  let \n1 = {rand*(#1)} in  +(0,\n1)

  \foreach \a in {0.0,0.1,...,#2}{
    let \n1 = {rnd*(#1)} in
    -- +(\a,\n1)
  }
}  % #1=seed, #2=length of horizontal line

\setlength{\parskip}{.5em}

% scrbook permite mudar - o tipo de documento que estou usando
% e corrigir o erro
%Package tocbasic Warning: number width of figure toc entries should be
\DeclareTOCStyleEntry[numwidth=45pt]{tocline}{figure}
\DeclareTOCStyleEntry[numwidth=45pt]{tocline}{section}
\DeclareTOCStyleEntry[numwidth=45pt]{tocline}{subsection}


% comentários nas margens com alinhamento que faz sentido
%https://tex.stackexchange.com/questions/411939/marginpar-text-alignment/417276#417276
\newcommand{\alignedmarginpar}[1]{%
    \Ifthispageodd{%
        \marginpar{\raggedright\small #1}}{%
        \marginpar{\raggedleft\small #1}}%
    }


% Marcas usando bbding ou outro pacote
\newcommand{\gxok}{\textcolor{green}{\CheckmarkBold}}
\newcommand{\gxnot}{\textcolor{red}{\XSolidBold}}


% https://tex.stackexchange.com/questions/32683/rotated-column-titles-in-tabular
% cria um comando para rodar os headings das tabelas
% e acertar o tamanho das colunas,
% parametros rotacao (op) tamanho(opcional) e texto (mandatório)
% precisa do xparse
\NewDocumentCommand{\gxrot}{O{90} O{1em} m}{\makebox[#2][l]{\rotatebox{#1}{#3}}}%


\newcommand{\gxibfcolor}{NavyBlue}
\newcommand{\gxibbcolor}{White}


\newenvironment{gxwinfobox}[2][r]%
{%
\begin{wrapfigure}{#1}{0.5\textwidth}%
\begin{tcolorbox}[title=\textbf{\Info #2},colframe=\gxibfcolor,colback=\gxibbcolor]%
}%
{%
\end{tcolorbox}%
\end{wrapfigure}%
}%


\newenvironment{gxinfobox}[1]
{%
\begin{tcolorbox}[title={\Large\Info\ }\textbf{#1},%
colframe=\gxibfcolor,colback=\gxibbcolor, sharp corners=uphill]%
}%
{%
\end{tcolorbox}%
}%


\newcommand{\gxcwinfobox}[3][r]{\begin{gxwinfobox}[#1]{#2}#3\end{gxwinfobox}}

\NewDocumentCommand{\whybox}{m+m}{
    \begin{tcolorbox}[title=\textbf{Por que #1?},colframe=blue!75!black,%

        colback=white,fonttitle=\bfseries,halign=center]%
        \protect#2
\end{tcolorbox}}

\newcommand{\gxatencao}[1]{
    \begin{tcolorbox}[notitle,boxrule=0mm,colframe=blue!15,%
        colback=blue!15,fonttitle=\bfseries,arc=0mm]%
        \centering
        \textbf{\protect#1}
\end{tcolorbox}}